\documentclass[12pt]{article}
\usepackage{graphicx} % Required for inserting images
\usepackage{amsmath}
\usepackage{amssymb}
\usepackage{amsthm}
\usepackage{xfrac}
\usepackage{mathtools}
\usepackage{relsize}
\usepackage{tikz}
\usepackage{tikz-cd} 
\usepackage{halloweenmath}
\usetikzlibrary{shapes.geometric}
\usepackage{parskip}
\usepackage{enumitem}

\makeatletter
\renewcommand*\env@matrix[1][*\c@MaxMatrixCols c]{%
  \hskip -\arraycolsep
  \let\@ifnextchar\new@ifnextchar
  \array{#1}}
\makeatother

\newcommand{\ops}{\mathcal{L}}
\newcommand{\reals}{\mathbb{R}}
\newcommand{\q}{\mathbb{Q}}
\newcommand{\nats}{\mathbb{N}}
\newcommand{\ints}{\mathbb{Z}}
\newcommand{\tr}{\text{tr}}
\newcommand{\spann}{\text{span}}
\newcommand{\complex}{\mathbb{C}}
\newcommand{\iprod}[2]{\langle #1, #2 \rangle}
\newcommand{\proj}[2]{\text{proj}_{#1}(#2)}
\newcommand{\makelarge}[1]{\mathlarger{\mathlarger{\mathlarger{#1}}}}
\newcommand{\interior}{\text{int}}
\newcommand{\boundary}{\text{bdry}}
\newcommand{\cover}{\mathcal{O}}
\newcommand{\g}[1]{\langle #1 \rangle}
\newcommand{\ba}{\mathfrak{B}}
\newcommand{\ord}{\text{ord}}
\newcommand{\lcm}{\text{lcm}}
\newcommand{\id}{\text{id}}
\newcommand{\pf}[2]{\dfrac{\partial #1}{\partial #2}}
\newcommand{\Aut}{\text{Aut}}
\newcommand{\Alt}{\text{Alt}}
\newcommand{\sgn}{\text{sgn}}
\newcommand{\grad}{\text{grad}}
\newcommand{\curl}{\text{curl}}
\newcommand{\divv}{\text{div}}
\newcommand{\intsmod}[1]{\ints / #1 \ints}
\newcommand{\bignorm}{\Bigg{|}\Bigg{|}}
\newcommand{\T}{\mathbb{T}}
\newcommand{\im}{\text{im}}
\newcommand{\coker}{\text{coker}}
\newcommand{\ind}{\text{index}}
\newcommand{\disk}{\mathbb{D}}

\newcommand{\lref}[1]{\textbf{Lemma~\ref{#1}}}
\newcommand{\tref}[1]{\textbf{Theorem~\ref{#1}}}
\newcommand{\pref}[1]{\textbf{Proposition~\ref{#1}}}

\newcommand{\mdef}[3]{\text{mdef}_{(#1, #2)}(#3)}

\usepackage[a4paper, total={6.1in, 8in}]{geometry}

\newtheorem{thm}{Theorem}
\newtheorem{lemma}{Lemma}
\newtheorem{prop}{Proposition}
\newtheorem{defin}{Definition}

\numberwithin{thm}{section}
\numberwithin{lemma}{section}
\numberwithin{prop}{section}
\numberwithin{defin}{section}

\author{Isaac Clark}
\title{Handout 3}

\begin{document}

\maketitle 

\section{The subspace topology}

Given $A \subseteq \reals^n$, we want to define a family of subsets of $A$, say $\tau_A$, such that, \begin{enumerate}
    \item $\varnothing, A \in \tau_A$.
    \item $\tau_A$ is closed under arbitrary unions and finite intersections.
    \item $(f \circ \iota)^{-1}(U) \in \tau_A$ whenever $U \subseteq \reals^m$ is open and $f: \reals^n \to \reals^m$ is continuous.
\end{enumerate}

Turns out that $\tau_A = \{ U \cap A \ | \ U \subseteq \reals^n \text{ open}\}$ is the unique family of sets with these properties. One may check, $\varnothing = \varnothing \cap A$ and $A = \reals^n \cap A$, and so $\varnothing, A \in \tau_A$. Further, \begin{align*}
    \bigcup_{i \in I} \big(U_i \cap A \big) &= \left( \bigcup_{i \in I}  U_i \right) \cap A \\
    (U \cap A) \cap (V \cap A) &= (U \cap V) \cap A
\end{align*} And so $\tau_A$ is closed under arbitrary unions and finite intersections. Finally, \begin{align*}
    \iota^{-1}(U) &= \{ x \in A \ | \ \iota(x) \in U \} \\
    &= \{ x \in \reals^n \ | \ x \in A \} \cap \{ x \in \reals^n \ | \ x \in U\} \\
    &= U \cap A \\ \\
    (f \circ \iota)^{-1}(U) &= \iota^{-1} (f^{-1}(U)) \\
    &= f^{-1}(U) \cap A
\end{align*} And so $(f\circ \iota)^{-1}(U)$ is open whever $U \subseteq \reals^m$ is open. I leave it as an exercise to adjust the definition of $\varepsilon - \delta$ and sequential continuity for $f$ defined on a subset of $\reals^n$, and to check that the definitions agree.

\begin{defin}
    We say that $V \subseteq A$ is open in $A$ if $V = U \cap A$ for some open $U \subseteq \reals^n$.
\end{defin}

\begin{defin}
    We say that $C \subseteq A$ is closed in $A$ if $A \setminus C$ is open in $A$.
\end{defin}

\begin{defin}
    We say that $f: A \to B$, where $A \subseteq \reals^n$ and $B \subseteq \reals^m$, if $f^{-1}(U)$ is open in $A$ whenever $U$ is open in $B$.
\end{defin}

\begin{defin}
    We say that $f: A \to B$ is a homeomorphism if $f$ is bijective, $f$ is continuous, and $f^{-1}$ is continuous.
\end{defin}

\begin{defin}
    We say that $A \subseteq \reals^n$ and $B \subseteq \reals^m$ are homeomorphic if there exists a homeomorphism $f: A \to B$. In this case, we write $A \cong B$.
\end{defin}

\begin{prop}
    Let $f: A \to B$ be bijective. Then, $f$ is open $\iff f^{-1}$ is continuous.
\end{prop}
\begin{proof}
    We observe that since $f$ is bijective, for each $y \in B$ there exists a unique $x_y \in A$ such that $f(x_y) = y$. So, \begin{equation*}
        (f^{-1})^{-1}(U) = \{ y \in B \ | \ f^{-1}(y) \in U \} = \{ f(x_y) \in B \ | \ x_y \in U \} = f(U)
    \end{equation*} And thus $(f^{-1})^{-1}(U)$ is open whenever $f(U)$ is open, and so we have the claim.
\end{proof}

\begin{prop}
    The following are homeomorphic to $(0,1)$, \begin{enumerate}[label=(\alph*)]
        \item $A = (a, b)$
        \item $B = (a, \infty)$
        \item $C = (-\infty, a)$
        \item $D = \reals$
    \end{enumerate}
\end{prop}
\begin{proof}
    For (a), put $f(x) = a + (b - a)x$.

    For (b), consider that $g(x) = a^{-1}x$ exhibits $(a, \infty) \cong (1, \infty)$. And that $h(x) = x^{-1}$ exhibits $(1, \infty) \cong (0, 1)$, and so $g \circ h$ is as desired.

    For (c), consider that $p(x) = -x + 2a$ exhibits $(-\infty, a) \cong (a, \infty)$, which is homeomorphic to $(0, 1)$ per (b), so the composition is as desired.

    For (d), consider that, per (a), $(0, 1) \cong (-\pi/2, \pi/2)$, and then that $q(x) = \tan(x)$ exhibits $(-\pi/2, \pi/2) \cong \reals$, and so the composition is as desired.
\end{proof}

\begin{prop}
    The relation $A \sim B \iff A \cong B$ is an equivalence relation.
\end{prop}
\begin{proof}
    $A \sim A$ via $f(x) = x$. If $A \sim B$ via $g$, then $B \sim A$ via $g^{-1}$. And the composition of homeomorphims is a homeomorphism.
\end{proof}

\begin{prop}
    $\{a_n\}_{n\in\nats} \subseteq A$ converges to $a \in A \iff$ for all $U \subseteq A$ which are open in $A$ and contain $a$, there exists some $N \in \nats$ such that $a_n \in U$ for all $n \geq N$.
\end{prop}
\begin{proof}
    Exercise. See PS4.
\end{proof}

\begin{thm}
    Suppose $f: X \to \reals^n$ is such that there are closed (in $X$) subsets $A, B \subseteq X$ such that $X = A \cup B$ and $f \circ \iota_A$ and $f \circ \iota_B$ are continuous. Then $f$ is continuous.
\end{thm}
\begin{proof}
    Let $C \subseteq \reals^n$ be closed. Then, since $f \circ \iota_A$ and $f \circ \iota_B$ are continuous, $f^{-1}(C) \cap A$ and $f^{-1}(C) \cap B$ are closed in $A$ and $B$ respectively. Since $A, B$ themselves are closed in $X$, $f^{-1}(C) \cap A$ and $f^{-1}(C) \cap B$ are closed in $X$. And so their union, $f^{-1}(C) = (f^{-1}(C) \cap A) \cup (f^{-1}(C) \cap B)$ is closed in $X$. Hence, $f$ is continuous.
\end{proof}

\begin{defin}
    We say that $A \subseteq \reals^n$ is discrete if $\tau_A = \mathcal{P}(A)$.
\end{defin}

\begin{prop}\label{DiscreteCharacterization}
    $A$ is discrete if and only if for all $a \in A$, there exists $U \subseteq \reals^n$ an open neighbourhood of $a$ such that $U \cap A = \{ a \}$.
\end{prop}
\begin{proof}
    We may first observe that $A$ is discrete if and only if $\{a \}$ is open in $A$ for all $a \in A$, as every subset of $A$ is the union of singletons. Then, $\{ a \}$ is open if and only if there exists some $U \subseteq\reals^n$ open such that $\{a\} = U \cap A$.
\end{proof}

\begin{prop}
    If $A \cong B$ and $A$ is discrete, then $B$ is discrete.
\end{prop}
\begin{proof}
    Let $f: A \to B$ be a homeomorphism. By \textbf{Proposition 1.1}, $f$ is open. For each $b \in B$, there exists some $a \in A$ such that $f(a) = b$. Then, $\{b \} = f(\{a\})$ is open in $B$. So by \pref{DiscreteCharacterization}, $B$ is discrete.
\end{proof}

\begin{prop}
    We have, \begin{enumerate}[label=(\arabic*)]
        \item $\ints$ is discrete.
        \item Finite sets are discrete.
        \item $\q$ is not discrete.
        \item $\{ 1/n \ | \ n \in \ints^+ \}$ is discrete.
        \item $\{ 1/n \ | \ n \in \ints^+ \} \cup \{0\}$ is not discrete.
    \end{enumerate}
\end{prop}
\begin{proof}
    We argue via \pref{DiscreteCharacterization}.

    For (1), $B_{1/2}(n) \cap \ints = \{ n \}$.

    For (2), let $F\subseteq\reals^n$ be finite, and pick $\varepsilon > 0$ such that $\varepsilon < 2 \min_{(x, y) \in F^2} ||x-y||$. Then $B_{\varepsilon} (x) \cap F = \{x\}$ for any $x \in F$.

    For (3), note that if $U, C \subseteq \reals$ are open and closed respectively, $U \setminus C$ is open. Thus, every open subset of $\reals$ intersects $\q$ at at least two points.

    For (4), simply observe that $(n + 2)^{-1} < (n+1)^{-1} < (n)^{-1}$ for all $n \in \ints^+$, and so adjacent points can be separated by open sets.

    For (5), by the Archimedian property, every open neighbourhood of $0$ contains $n^{-1}$ for some $n \in \ints^+$.
\end{proof}

\begin{prop}
    If $A$ is discrete, then $A$ is countable.
\end{prop}
\begin{proof}
    Since $\q^{n+1}$ is countable, we may enumerate the set of all open balls with centers in $\q^n$ and rational radii, say by $j \mapsto B_j$. Then, for each $a \in A$, by \pref{DiscreteCharacterization} there exists some $U \subseteq \reals^n$ open such that $A \cap U = \{a \}$. But, by density we can find some $j$ such that $a \in B_j \subseteq U$. So we define a map $A \to \nats$ which sends every point to the least $j \in \nats$ such that $B_j$ is as desired, which exhibits an injection from $A$ to $\nats$, whence $A$ is countable.
\end{proof}

\begin{prop}
    If $f: X \to Y$ is a homeomorphism and $A \subseteq X$, then $f: A \to f(A)$ is a homeomorphism.
\end{prop}
\begin{proof}
    Exercise. See PS4.
\end{proof}

\begin{prop}
    $f: \reals^n \to \reals^m$ is continuous if and only if $f \circ \iota_A$ is continuous for all $\varnothing \ne A \subsetneq X$.
\end{prop}
\begin{proof}
    See PS4.
\end{proof}

\begin{prop}
    If $f: A \to B$ is continuous and $B \subseteq \reals^m$, then $f: A \to \reals^m$ is continuous.
\end{prop}
\begin{proof}
    See PS4.
\end{proof}

\end{document}