\documentclass[12pt]{article}
\usepackage{graphicx} % Required for inserting images
\usepackage{amsmath}
\usepackage{amssymb}
\usepackage{amsthm}
\usepackage{xfrac}
\usepackage{mathtools}
\usepackage{relsize}
\usepackage{tikz}
\usepackage{tikz-cd} 
\usepackage{halloweenmath}
\usetikzlibrary{shapes.geometric}
\usepackage{parskip}
\usepackage{enumitem}

\makeatletter
\renewcommand*\env@matrix[1][*\c@MaxMatrixCols c]{%
  \hskip -\arraycolsep
  \let\@ifnextchar\new@ifnextchar
  \array{#1}}
\makeatother

\newcommand{\ops}{\mathcal{L}}
\newcommand{\reals}{\mathbb{R}}
\newcommand{\q}{\mathbb{Q}}
\newcommand{\nats}{\mathbb{N}}
\newcommand{\ints}{\mathbb{Z}}
\newcommand{\tr}{\text{tr}}
\newcommand{\spann}{\text{span}}
\newcommand{\complex}{\mathbb{C}}
\newcommand{\iprod}[2]{\langle #1, #2 \rangle}
\newcommand{\proj}[2]{\text{proj}_{#1}(#2)}
\newcommand{\makelarge}[1]{\mathlarger{\mathlarger{\mathlarger{#1}}}}
\newcommand{\interior}{\text{int}}
\newcommand{\boundary}{\text{bdry}}
\newcommand{\cover}{\mathcal{O}}
\newcommand{\g}[1]{\langle #1 \rangle}
\newcommand{\ba}{\mathfrak{B}}
\newcommand{\ord}{\text{ord}}
\newcommand{\lcm}{\text{lcm}}
\newcommand{\id}{\text{id}}
\newcommand{\pf}[2]{\dfrac{\partial #1}{\partial #2}}
\newcommand{\Aut}{\text{Aut}}
\newcommand{\Alt}{\text{Alt}}
\newcommand{\sgn}{\text{sgn}}
\newcommand{\grad}{\text{grad}}
\newcommand{\curl}{\text{curl}}
\newcommand{\divv}{\text{div}}
\newcommand{\intsmod}[1]{\ints / #1 \ints}
\newcommand{\bignorm}{\Bigg{|}\Bigg{|}}
\newcommand{\T}{\mathbb{T}}
\newcommand{\im}{\text{im}}
\newcommand{\coker}{\text{coker}}
\newcommand{\ind}{\text{index}}

\newcommand{\lref}[1]{\textbf{Lemma~\ref{#1}}}
\newcommand{\tref}[1]{\textbf{Theorem~\ref{#1}}}
\newcommand{\pref}[1]{\textbf{Proposition~\ref{#1}}}

\newcommand{\mdef}[3]{\text{mdef}_{(#1, #2)}(#3)}

\usepackage[a4paper, total={6.1in, 8in}]{geometry}

\newtheorem{thm}{Theorem}
\newtheorem{lemma}{Lemma}
\newtheorem{prop}{Proposition}
\newtheorem{defin}{Definition}

\numberwithin{thm}{section}
\numberwithin{lemma}{section}
\numberwithin{prop}{section}
\numberwithin{defin}{section}

\author{Isaac Clark}
\title{Handout 4}

\begin{document}

\maketitle

\section{Connectedness}

\begin{defin}
    We say that $\varnothing \ne X \subseteq \reals^n$ is disconnected if there exist $\varnothing \ne A, B \subseteq X$ disjoint, open (in $X$) sets such that $X = A \cup B$. Such $A, B$ are called a disconnection of $X$. Equivalently, $X$ is disconnected if there exists a proper clopen subset of $X$.
\end{defin}

\begin{defin}
    We say that $\varnothing \ne X \subseteq \reals^n$ is connected if it is not disconnected.
\end{defin}

\begin{defin}
    We say that $\varnothing \ne X \subseteq \reals^n$ is path-connected if for every $x, x' \in X$, there exists a continuous function $f: [0, 1] \to X$ such that $f(0) = x$ and $f(1) = x'$. Such an $f$ is called a path from $x$ to $x'$ in $X$.
\end{defin}

\begin{thm}\label{[0,1]IsConnected}
    $[0, 1]$ is connected.
\end{thm}
\begin{proof}
    Suppose, for a contradiction, that $[0, 1]$ is disconnected. Let $A, B \subseteq [0,1]$ be a disconnection of $[0,1]$. We may assume without loss of generality that $0 \in A$. Let $s = \sup A$ and observe that $0 \leq s \leq 1$. So, since $[0, 1] = A \cup B$, either $s \in A$ or $s \in B$. If $s \in A$, then since $A$ is open, there exists some $\varepsilon > 0$ such that $(s - \varepsilon , s + \varepsilon) \subseteq A$. But then $s + \varepsilon/2 \in A$ with $s < s + \varepsilon/2$, contradicting that $s$ is an upper bound of $A$. If $s \in B$, then, since $B$ is open, there exists some $\varepsilon > 0$ such that $(s - \varepsilon , s + \varepsilon) \subseteq B$. But, since $s \in \overline{A}$, there must be some point $a \in A$ such that $a \in (s - \varepsilon, s + \varepsilon)$, contradicting that $A$ and $B$ are disjoint. Hence, $[0, 1]$ is connected.
\end{proof}

\begin{thm}\label{ContinuousImageOfConnectedIsConnected}
    Suppose $X \subseteq \reals^n$ is (path-)connected and $f: X \to \reals^m$ is continuous. Then $f(X)$ is (path-)connected.
\end{thm}
\begin{proof}
    Suppose $X$ is path-connected. Given $y, y' \in f(X)$, pick $x, x'$ such that $f(x) = y$ and $f(x') = y'$. Let $\gamma: [0, 1] \to X$ be a path connecting $x$ and $x'$ in $X$. Then, $\gamma' = f \circ \gamma : [0, 1] \to f(X)$ is continuous and satisifes $\gamma'(0) = f(\gamma(0)) = f(x) = y$ and $\gamma'(1) = y'$. Thus, there is a path from $y$ to $y'$ for any $y,y' \in f(X)$, and so $f(X)$ is path-connected.

    For the second part, we will show the contrapositive. Suppose $f(X)$ is disconnected. Let $A, B$ be a disconnection of $f(X)$. Then, consider $f^{-1}(A)$ and $f^{-1}(B)$. Each are open, since $f$ is continuous, they are disjoint, since $A, B$ are disjoint, and their union is $X$, and so they exhibit a disconnection of $X$.
\end{proof}

\begin{prop}\label{DisconnectionCharacterization}
    $X$ is disconnected $\iff$ there exists a non-constant, continuous function $f: X \to \{0, 1\}$. Equivalently, $X$ is connected $\iff$ every continuous function $f: X \to \{0, 1\}$ is constant.
\end{prop}
\begin{proof}
    If $f: X \to \{0, 1\}$ is non-constant and continuous, then $f^{-1}(\{0\})$ and $f^{-1}(\{1\})$ are a disconnection of $X$. If $X$ is disconnected, let $A, B$ be a disconnection of $X$, and define $f(x) = \chi_A(x)$, i.e. $f(x) = 1$ if $x \in A$ and $f(x) = 0$ otherwise. Then, one may check that the only open sets in $\{0, 1\}$ are $\varnothing, \{0\} , \{1\}, \{0,1\}$, each of which are, by construction, open in $X$. Since $A \ne X$, $f$ is not constant, and so is as desired.
\end{proof}

\begin{prop}\label{UnionOfConnected}
    Let $\{ X_i \}_{i \in I}$ be a family of (path-)connected sets. Then, if we further have that $\bigcap_{i \in I} X_i \ne \varnothing$ then $\bigcup_{i \in I} X_i$ is (path-)connected.
\end{prop}
\begin{proof}
    Pick some $x \in \bigcap_{i \in I} X_i$ and let $C \subseteq \bigcup_{i \in I} X_i$ be a clopen neighbourhood of $x$. Each $X_i \cap C$ is clopen and non-empty in $X_i$ and so, since each $X_i$ is connected, $X_i \subseteq C$ for every $i \in I$. So, $\bigcup_{i \in I} X_i \subseteq C$, and so we have equality and thus that $\bigcup_{i \in I} X_i$ contains no non-trivial clopen sets, and so is connected.
\end{proof}

\begin{thm}\label{PathConnectedImpliesConnected}
    If $X$ is path-connected, then it is also connected.
\end{thm}
\begin{proof}
    Fix $x_0 \in X$. Then, for each $x \in X$, let $\gamma_x: [0,1]\to X$ be a path from $x_0$ to $x$ in $X$. Then, $\bigcup_{x \in X} \gamma_x([0,1]) \subseteq X$ since each image is contained in $X$. Also, $X \subseteq \gamma_x([0,1])$ since $x \in \gamma_x([0, 1])$ for each $x \in X$. Further, $x_0 \in \gamma_x([0,1])$ for every $x \in X$, and so $\bigcap_{x \in X} \gamma_x ([0,1]) \ne \varnothing$, and, since each $\gamma_x$ is continuous by \tref{ContinuousImageOfConnectedIsConnected} and $[0,1]$ is connected by \tref{[0,1]IsConnected}, $\gamma_x([0,1])$ is connected for each $x \in X$. Taken all together, by \pref{UnionOfConnected}, $X = \bigcup_{x \in X} \gamma_x([0,1])$ is connected.
\end{proof}

\begin{thm}
    There exists $X \subseteq \reals^2$ such that $X$ is connected but not path-connected.
\end{thm}
\begin{proof}
    Let $X = \{ (x, \sin (1/x)) \ | \ x > 0 \} \cup \{ (0, 0) \}$. The rest is an exercise. \textit{Hint:} Let $f: [0, 1] \to X$ be any continuous function with $f(0) = (0, 0)$ and consider the $\sup \{ t \in [0, 1] \ | \ f([0, t]) = \{(0, 0)\} \ \}$.
\end{proof}

\begin{thm}\label{ProductOfConnected}
    If $X, Y$ are (path-)connected then $X \times Y$ is (path-)connected.
\end{thm}
\begin{proof}
    Fix $y \in Y$. For $x \in X$, let $U_x = \big( \{x\} \times Y \big) \cup \big( X \times \{y\} \big)$. Note that each $U_x$ is connected by \tref{ContinuousImageOfConnectedIsConnected}, since it is the image of $Y, X$ under $y' \mapsto (x, y')$ and $x' \mapsto (x', y)$ respectively, and $(x, y) \in \big(\{x\} \times Y \big) \cap \big( X \times \{y\} \big)$ so we conclude by \pref{UnionOfConnected}. Clearly $X \times Y = \bigcup_{x \in X} U_x$ and $\varnothing \ne X \times \{y\} \subseteq \bigcap_{x \in X} U_x$, so again by \pref{UnionOfConnected}, $X \times Y$ is connected.
\end{proof}

\begin{thm}\label{ClosureOfConnected}
    If $X$ is connected and $X \subseteq Y \subseteq \overline{X}$ then $Y$ is connected.
\end{thm}
\begin{proof}
    Let $f: Y \to \{0, 1\}$ be a continuous function. Then $f|_X = f \circ \iota_X$ is continuous, and so, since $X$ is connected, by \pref{DisconnectionCharacterization}, it is constant. Since $Y \subseteq \overline{X}$, we have that $X$ is dense in $Y$, and so that $f$ is also constant, whence, again by \pref{DisconnectionCharacterization}, $Y$ is disconnected.
\end{proof}

\begin{prop}
    The following are (path-)connected: \begin{enumerate}
        \item Intervals (i.e. $(a, b), [a, b)$, etc.).
        \item Rays (i.e. $(a, \infty)$, etc.).
        \item $\reals^n$.
        \item $B_r(x)$.
    \end{enumerate}
\end{prop}
\begin{proof}
    For (1), by \tref{ContinuousImageOfConnectedIsConnected} and \tref{[0,1]IsConnected} we have that any interval of the form $[a, b]$ is connected. Then, $(0, 1] = \bigcup_{n \in \ints^+} [1/n, 1]$ and so, by \pref{UnionOfConnected}, it is connected. We may construct each other unit interval type in a similar manner, and then use \tref{ContinuousImageOfConnectedIsConnected}.

    For (2), pick $N > a$, then $(a, \infty) = \bigcup_{n \geq N} (a, n)$, and so we may apply \pref{UnionOfConnected}. The other cases follow similarly.

    For (3), write $\reals = \bigcup_{n \in \ints^+} (-n, n)$ and use \pref{UnionOfConnected}. Then proceed by induction and use \tref{ProductOfConnected}.

    For (4), observe that if $a, b \in B_r(0)$ then $t \mapsto (1 - t)a + tb$ exhibits a path from $a$ to $b$ in $B_r(0)$, so by \tref{PathConnectedImpliesConnected}, it is connected. Then, $B_r(x)$ is simply the image of $B_r(0)$ under $y \mapsto y + x$, so apply \tref{ContinuousImageOfConnectedIsConnected}.
\end{proof}

\begin{defin}
    We say that $\varnothing \ne X \subseteq \reals^n$ is totally disconnected if the only connected subsets of $X$ are the singletons.
\end{defin} \newpage

\begin{prop}
    The following are totally disconnected: \begin{enumerate}
        \item Discrete sets.
        \item $\q$.
        \item The Cantor set, see Handout 1.
    \end{enumerate}
\end{prop}
\begin{proof}
    For (1), simply observe that discrete sets have only isolated points, so each may be covered by some $B_{\varepsilon}(x)$. Then, if the set is a singleton, it is clearly connected. If the set contains more than 1 point, fix one, and consider the disconnection given by the ball about the fixed point and the union of the balls about each other point.

    (2) follows from the density of $\reals \setminus \q$.

    For (3), the idea is that for any two distinct points in the Cantor set will eventually belong to different sub-intervals, and so these respective closed intervals exhibit a disconnection. \textit{Exercise:} Make this formal.
\end{proof}

\begin{defin}
    We say that $C \subseteq \reals^n$ is convex if for all $a, b \in C$, we have that $(1 - t)a + tb \in C$ for every $t \in [0, 1]$.
\end{defin}

\begin{thm}
    Suppose $X \subseteq \reals$. Then the following are equivalent, \begin{enumerate}
        \item $X$ is connected.
        \item $X$ is convex.
        \item $X$ is path-connected.
    \end{enumerate}
\end{thm}
\begin{proof}
    (1) $\implies$ (2). Suppose $a, b \in X$ and $a < t < b$. Then, $X \cap (- \infty, t)$ and $X \cap (t, \infty)$ are open in $X$ and so, if $t \not\in X$ then this would exhibit a disconnection of $X$, which would be a contradiction, hence $t \in X$, and so $X$ is convex.

    (2) $\implies$ (3). $t \mapsto (1 - t)a + tb$ is a path from $a$ to $b$ in $X$ by definition.

    (3) $\implies$ (1). This is \tref{PathConnectedImpliesConnected}.
\end{proof}

\textbf{Corollary}. If $f: X \to \reals$ is continuous and $X$ is connected then $f$ attains all values between any two points in the image, i.e. the Intermediate Value theorem.

\end{document}